\part{Additional Topics}


\chapter{Point Estimation}\label{mle.chapter}

\noindent
This chapter defines the workhorses of non-Bayesian estimation,
maximum likelihood and penalized maximum likelihood, and relates them
to Bayesian point estimation based on posterior means, medians, and
modes.  Such estimates are called ``point estimates'' because they
are composed of a single value for the model parameters $\theta$
rather than a posterior distribution.

Stan's optimizer can be used to implement (penalized) maximum
likelihood estimation for any likelihood function and penalty function
that can be coded in Stan's modeling language.  Stan's optimizer can
also be used for point estimation in Bayesian settings based on
posterior modes.  Stan's Markov chain Monte Carlo samplers can be used
to implement point inference in Bayesian models based on posterior
means or medians.

\section{Maximum Likelihood Estimation}\label{mle.section}

Given a likelihood function $p(y|\theta)$ and a fixed data vector $y$,
the maximum likelihood estimate (MLE) is the parameter vector $\hat{\theta}$
that maximizes the likelihood, i.e., 
\[
\hat{\theta} = \mbox{argmax}_{\theta} \ p(y|\theta).
\]
It is usually more convenient to work on the log scale.  
An equivalent%
%
\footnote{The equivalence follows from the fact that densities are
  positive and the log function is strictly monotonic, i.e., 
  $p(y|\theta) \geq 0$ and for all $a, b > 0$, $\log a > \log b$ if and
  only if $a > b$.}
%
formulation of the MLE is
%
\[
\hat{\theta} = \mbox{argmax}_{\theta} \ \log p(y|\theta).
\]

\subsection{Existence of Maximum Likelihood Estimates}

Because not all functions have unique maximum values, maximum
likelihood estimates are not guaranteed to exist.  As discussed in
\refchapter{problematic-posteriors}, this situation can arise when
%
\begin{itemize}
\item there is more than one point that maximizes the likelihood function,
\item the likelihood function is unbounded, or
\item the likelihood function is bounded by an asymptote that is never
  reached for legal parameter values.
\end{itemize}
%
These problems persist with the penalized maximum likelihood estimates
discussed in the next section, and Bayesian posterior modes as
discussed in the following section.  


\subsection{Example: Linear Regression}

Consider an ordinary linear regression problem with an $N$-dimensional
vector of observations $y$, an $(N \times K)$-dimensional data matrix
$x$ of predictors, a $K$-dimensional parameter vector $\beta$ of
regression coefficients, and a real-valued noise scale $\sigma > 0$,
with log likelihood function
\[
\log p(y|\beta,x) = \sum_{n=1}^N \log \distro{Normal}(y_n|x_n \beta,
\sigma).
\]
%
The maximum likelihood estimate for $\theta = (\beta,\sigma)$ is just
\[
(\hat{\beta},\hat{\sigma})
\ = \
\mbox{argmax}_{\beta,\sigma} 
\log p(y|\beta,\sigma,x) = \sum_{n=1}^N \log \distro{Normal}(y_n|x_n \beta, \sigma).
\]

\subsubsection{Squared Error}

A little algebra on the log likelihood function shows that the
marginal maximum likelihood estimate $\hat{\theta} =
(\hat{\beta},\hat{\sigma})$ can be equivalently formulated for
$\hat{\beta}$ in terms of least squares.  That is, $\hat{\beta}$ is
the value for the coefficient vector that minimizes the sum of squared
prediction errors,
%
\[
\hat{\beta} 
\ = \
\mbox{argmin}_{\beta} \sum_{n=1}^N (y_n - x_n \beta)^2
\ = \ 
\mbox{argmin}_{\beta} (y - x \beta)^{\top} (y - x\beta).
\]
%
The residual error for data item $n$ is the difference between the
actual value and predicted value, $y_n - x_n \hat{\beta}$.  The
maximum likelihood estimate for the noise scale, $\hat{\sigma}$ is
just the square root of the average squared residual, 
\[
\hat{\sigma}^2
\ = \
\frac{1}{N} \sum_{n=1}^N \left( y_n - x_n \hat{\beta} \right)^2
\ = \
\frac{1}{N} (y - x \hat{\beta})^{\top} (y - x\hat{\beta}).
\]

\subsubsection{Minimizing Squared Error in Stan}

The squared error approach to linear regression can be directly coded
in Stan with the following model.
%
\begin{stancode}
data {
  int<lower=0> N;
  int<lower=1> K;
  vector[N] y;
  matrix[N,K] x;
}
parameters {
  vector[K] beta;
}
transformed parameters {
  real<lower=0> squared_error;
  squared_error <- dot_self(y - x * beta);
}
model {
  increment_log_prob(-squared_error);
}  
generated quantities {
  real<lower=0> sigma_squared;
  sigma_squared <- squared_error / N;
}
\end{stancode}
%
Running Stan's optimizer on this model produces the MLE for the linear
regression by directly minimizing the sum of squared errors and using
that to define the noise scale as a generated quantity.

By replacing \code{N} with \code{N-1} in the denominator of the
definition of \code{sigma\_squared}, the more commonly supplied
unbiased estimate of $\sigma^2$ can be calculated; see
\refsection{estimation-bias} for a definition of estimation bias and a
discussion of estimating variance.



\section{Penalized Maximum Likelihood Estimation}

There is nothing special about a likelihood function as far as the
ability to perform optimization is concerned.  It is common among
non-Bayesian statisticians to add so-called ``penalty'' functions
to log likelihoods and optimize the new function.  The penalized
maximum likelihood estimator for a log likelihood function 
$\log p(y|\theta)$ and penalty function $r(\theta)$ is defined to be
\[
\hat{\theta} = \mbox{argmax}_{\theta} \log p(y|\theta) - r(\theta).
\]
The penalty function $r(\theta)$ is negated in the maximization so
that the estimate $\hat{\theta}$ balances maximizing the log
likelihood and minimizing the penalty.  Penalization is sometimes
called ``regularization.''


\subsection{Examples}\label{penalized-mle-examples}

\subsubsection{Ridge Regression}

Ridge regression \citep{HoerlKennard:1970} is based on penalizing the
Euclidean length of the coefficient vector $\beta$. The ridge penalty
function is
%
\[
r(\beta) 
\ = \
\lambda \, \sum_{k=1}^K \beta_k^2
\ = \ 
\lambda \, \beta^{\top} \beta,
\]
%
where $\lambda$ is a constant tuning parameter that determines the
magnitude of the penalty. 


Therefore, the penalized maximum likelihood estimate for ridge
regression is just
%
\[
(\hat{\beta},\hat{\sigma}) 
\ = \
\mbox{argmax}_{\beta,\sigma} \,
 \sum_{n=1}^N \log \distro{Normal}(y_n|x_n \beta, \sigma) - \lambda
 \sum_{k=1}^K \beta_k^2
\]
%
The ridge penalty is sometimes called L2 regularization or shrinkage,
because of its relation to the L2 norm.

Like the basic MLE for linear regression, the ridge regression
estimate for the coefficients $\beta$ can also be formulated in terms
of least squares,
%
\[
\hat{\beta}
\ = \ 
\mbox{argmin}_{\beta} \, \sum_{n=1}^N (y_n - x_n \beta)^2 + \sum_{k=1}^K \beta_k^2
\ = \ 
\mbox{argmin}_{\beta} \, (y - x\beta)^{\top} (y - x\beta) +
\lambda \beta^{\top} \beta.
\]

The effect of adding the ridge penalty function is that the ridge
regression estimate for $\beta$ is a vector of shorter length, or in
other words, $\hat{\beta}$ is shrunk.  The ridge estimate does not
necessarily have a smaller absolute $\beta_k$ for each $k$, nor does
the coefficient vector necessarily point in the same direction as the
maximum likelihood estimate.

In Stan, adding the ridge penalty involves adding its magnitude as a
data variable and the penalty itself to the model block,
%
\begin{stancode}
data {
  // ...
  real<lower=0> lambda;
}
// ...
model {
  // ...
  increment_log_prob(- lambda * dot_self(beta));
}
\end{stancode}
%
The noise term calculation remains the same.

\subsubsection{The Lasso}

The lasso \citep{Tibshirani:1996} is an alternative to ridge
regression that applies a penalty based on the sum of the absolute
coefficients, rather than the sum of their squares,
\[
r(\beta) = \lambda \sum_{k=1}^K | \beta_k |.
\]
The lasso is also called L1 shrinkage due to its relation to the L1
norm, which is also known as taxicab distance or Manattan distance.

Because the derivative of the penalty does not depend on the value of
the $\beta_k$,
\[
\frac{d}{d\beta_k} \lambda \sum_{k=1}^K | \beta_k | =
\mbox{signum}(\beta_k),
\]
it has the effect of shrinking parameters all the way to 0 in maximum
likelihood estimates.  Thus it can be used for variable selection as
well as just shrinkage.%
%
\footnote{In practice, Stan's gradient-based optimizers are not
  guaranteed to produce exact zero values; see
  \cite{LangfordEtAl:2009} for a discussion of getting exactly zero
  values with gradient descent.}
%
The lasso can be implemented in Stan just as easily as ridge
regression, with the magnitude declared as data and the penalty added
to the model block,
%
\begin{stancode}
data {
  // ...
  real<lower=0> lambda;
}
// ...
model {
  // ...
  for (k in 1:K)
    increment_log_prob(- lambda * abs(beta[k]));
}
\end{stancode}

\subsubsection{The Elastic Net}

The naive elastic net \citep{ZouHastie:2005} involves a weighted
average of ridge and lasso penalties, with a penalty function
\[
r(\beta)
= \lambda_1 \sum_{k=1}^K |\beta_k|
+ \lambda_2 \sum_{k=1}^K \beta_k^2.
\]
The naive elastic net combines properties of both ridge regression and
the lasso, providing both identification and variable selection.

The naive elastic net can be implemented directly in Stan by combining
implementations of ridge regression and the lasso, as
%
\begin{stancode}
data {
  real<lower=0> lambda1;
  real<lower=0> lambda2;
  // ...
}
// ...
model {
  // ...
  for (k in 1:K)
    increment_log_prob(-lambda1 * fabs(beta[k]));
  increment_log_prob(-lambda2 * dot_self(beta));
}
\end{stancode}
%
Note that the signs are negative in the program because $r(\beta)$ is
a penalty function.

The elastic net \citep{ZouHastie:2005} involves adjusting the final estimate for
$\beta$ based on the fit $\hat{\beta}$ produced by the naive elastic
net.  The elastic net estimate is
\[
\hat{\beta} = (1 + \lambda_2) \beta^*
\]
where $\beta^{*}$ is the naive elastic net estimate.

To implement the elastic net in Stan, the data, parameter, and model
blocks are the same as for the naive elastic net.  In addition, the
elastic net estimate is calculated in the generated quantities block.
%
\begin{stancode}
generated quantities {
  vector[K] beta_elastic_net;
  // ...
  beta_elastic_net <- (1 + lambda2) * beta;
}
\end{stancode}
%
The error scale also needs to be calculated in the generated
quantities block based on the elastic net coefficients
\code{beta\_elastic\_net}.


\subsubsection{Other Penalized Regressions}

It is also common to use penalty functions that bias the coefficient
estimates toward values other than 0, as in the estimators of
\cite{JamesStein:1961}.  Penalty functions can also be used to bias
estimates toward population means; see
\citep{EfronMorris:1975,Efron:2012}.  This latter approach is similar
to the hierarchical models commonly employed in Bayesian statistics.


\section{Posterior Mode Estimation}

There are three common approaches to Bayesian point estimation based
on the posterior $p(\theta|y)$ of parameters $\theta$ given observed
data $y$: the mode (maximum), the mean, and the median.  This section
covers estimates based on the parameters $\theta$ that maximize the
posterior density, and the next sections continue with discussions of
the mean and median.

An estimate based on a model's posterior mode can be defined by
%
\[
\hat{\theta} = \mbox{argmax}_{\theta} \, p(\theta|y).
\]
%
When it exists, $\hat{\theta}$ maximizes the posterior density of the
parameters given the data.  The posterior mode is sometimes called the
``maximum a posteriori'' (MAP) estimate.

As discussed in \refchapter{problematic-posteriors} and
\refsection{mle}, a unique posterior mode might not
exist---there may be no value that maximizes the posterior mode or
there may be more than one.  In these cases, the posterior mode
estimate is undefined.  Stan's optimizer, like most optimizers, will
have problems in these situations.  It may also return a locally
maximal value that is not the global maximum.

In cases where there is a posterior mode, it will correspond to a
penalized maximum likelihood estimate with a penalty function equal to
the negation of the log prior.  This is because Bayes's rule,
\[
p(\theta|y) = \frac{p(y|\theta) \, p(\theta)}{p(y)},
\]
ensures that
%
\begin{eqnarray*}
\mbox{argmax}_{\theta} \ p(\theta|y)
& = &
\mbox{argmax}_{\theta} \ \frac{p(y|\theta) \, p(\theta)}{p(y)}
\\[6pt]
& = &
\mbox{argmax}_{\theta} \ p(y|\theta) \, p(\theta),
\end{eqnarray*}
%
and the positiveness of densities and the strict monotonicity of log
ensure that
\[
\mbox{argmax}_{\theta} \ p(y|\theta) \, p(\theta)
\ = \
\mbox{argmax}_{\theta} \ \log p(y|\theta) + \log p(\theta).
\]
%

In the case where the prior (proper or improper) is uniform, the
posterior mode is equivalent to the maximum likelihood estimate.

For most commonly used penalty functions, there are probabilistic
equivalents.  For example, the ridge penalty function corresponds to a
normal prior on coefficients and the lasso to a Laplace prior.  The
reverse is always true---a negative prior can always be treated as a
penalty function.



\section{Posterior Mean Estimation}

A standard Bayesian approach to point estimation is to use the
posterior mean (assuming it exists), defined by
%
\[
\hat{\theta} = \int \theta \, p(\theta|y) \, d\theta.
\]
%
The posterior mean is often called {\it the}\ Bayesian estimator,
because it's the estimator that minimizes the expected square error of
the estimate.

An estimate of the posterior mean for each parameter is returned by
Stan's interfaces;  see the RStan, CmdStan, and PyStan user's guides
for details on the interfaces and data formats.  

Posterior means exist in many situations where posterior
modes do not exist.  For example, in the $\distro{Beta}(0.1, 0.1)$
case, there is no posterior mode, but posterior mean is well defined
with value 0.5.  

A situation where posterior means fail to exist but posterior modes do
exist is with a posterior with a Cauchy distribution
$\distro{Cauchy}(\mu,\tau)$.  The posterior mode is $\mu$, but the
integral expressing the posterior mean diverges.  Such diffuse priors
rarely arise in practical modeling applications; even with a Cauchy
Cauchy prior for some parameters, data will provide enough constraints
that the posterior is better behaved and means exist.

Sometimes when posterior means exist, they are not meaningful, as in
the case of a multimodal posterior arising from a mixture model or in
the case of a uniform distribution on a closed interval.  


\section{Posterior Median Estimation}

The posterior median (i.e., 50th percentile or 0.5 quantile) is
another popular point estimate reported for Bayesian models.  The
posterior median minimizes the expected absolute error of estimates.
These estimates are returned in the various Stan interfaces;  see the
RStan, PyStan and CmdStan user's guides for more information on
format. 

Although posterior medians may fail to be meaningful, they often exist
even where posterior means do not, as in the Cauchy distribution.



\section{Estimation Error, Bias, and Variance}\label{estimation-bias.section}

An estimate $\hat{\theta}$ depends on the particular data $y$ and
either the log likelihood function, $\log p(y|\theta)$, penalized log
likelihood function $\log p(y|\theta) - r(\theta)$, or log probability
function $\log p(y,\theta) = \log p(y,\theta) + \log p(\theta)$.  In
this section, the notation $\hat{\theta}$ is overloaded to indicate
the estimator, which is an implicit function of the data and
(penalized) likelihood or probability function.

\subsection{Estimation Error}

For a particular observed data set $y$ generated according to true
parameters $\theta$, the estimation error is the difference between
the estimated value and true value of the parameter,
\[
\mbox{err}(\hat{\theta}) = \hat{\theta} - \theta.
\]
%

\subsection{Estimation Bias}

For a particular true parameter value $\theta$ and a likelihood
function $p(y|\theta)$, the expected estimation error averaged over
possible data sets $y$ according to their density under the likelihood
is
%
\[
\mathbb{E}_{p(y|\theta)}[\hat{\theta}] 
\ = \
\int \left( \mbox{argmax}_{\theta'} p(y|\theta') \right) p(y|\theta) dy.
\]

An estimator's bias is the expected estimation error,
%
\[
\mathbb{E}_{p(y|\theta)}[\hat{\theta} - \theta]
\ = \
\mathbb{E}_{p(y|\theta)}[\hat{\theta}] - \theta
\]
%
The bias is a multivariate quantity with the same dimensions as
$\theta$.  An estimator is unbiased if its expected estimation error
is zero and biased otherwise.

\subsubsection{Example: Estimating a Normal Distribution}

Suppose a data set of observations $y_n$ for $n \in 1{:}N$ drawn from
a normal distribution.  This presupposes a model $y_n \sim
\distro{Normal}(\mu,\sigma)$, where both $\mu$ and $\sigma > 0$ are
parameters.  The log likelihood is just
\[
\log p(y|\mu,\sigma) = \sum_{n=1}^N \log
\distro{Normal}(y_n|\mu,\sigma).
\]
The maximum likelihood estimator for $\mu$ is just the sample mean,
i.e., the average of the samples,
\[
\hat{\mu} = \frac{1}{N} \sum_{n=1}^N y_n.
\]
The maximum likelihood estimate for the mean is unbiased.

The maximum likelihood estimator for the variance $\sigma^2$ is the
average of the squared difference from the mean,
\[
\hat{\sigma}^2 = \frac{1}{N} \sum_{n=1}^N (y_n - \hat{\mu})^2.
\]
The maximum likelihood for the variance is biased on the low side,
i.e., 
%
\[
\mathbb{E}_{p(y|\mu,\sigma)}[\hat{\sigma}^2] < \sigma.
\]
%
The reason for this bias is that the maximum likelihood estimate is
based on the difference from the estimated mean $\hat{\mu}$.  Plugging
in the actual mean can lead to larger sum of squared differences;  if
$\mu \neq \hat{\mu}$, then
\[
\frac{1}{N} \sum_{n=1}^N (y_n - \mu)^2
> 
\frac{1}{N} \sum_{n=1}^N (y_n - \hat{\mu})^2.
\]

An alternative estimate for the variance is the sample variance, which
is defined by
\[
\hat{\mu} = \frac{1}{N-1} \sum_{n=1}^N (y_n - \hat{\mu})^2.
\]
This value is larger than the maximum likelihood estimate by a factor
of $N/(N-1)$.


\subsection{Estimation Variance}

The variance of component $k$ of an estimator $\hat{\theta}$ is
computed like any other variance, as the expected squared difference
from its expectation, 
%
\[
\mbox{var}_{p(y|\theta})[\hat{\theta}_k]
\ = \
\mathbb{E}_{p(y|\theta})[\, (\hat{\theta}_k -
\mathbb{E}_{p(y|\theta)}[\hat{\theta}_k])^2 \,].
\]
%
The full $K \times K$ covariance matrix for the estimator is thus
defined, as usual, by
%
\[
\mbox{covar}_{p(y|\theta)}[\hat{\theta}]
\ = \
\mathbb{E}_{p(y|\theta})[\, (\hat{\theta} - \mathbb{E}[\hat{\theta}]) \,
                         (\hat{\theta} -
                         \mathbb{E}[\hat{\theta}])^{\top} \, ].
\]

Continuing the example of estimating the mean and variance of a normal
distribution based on sample data, the maximum likelihood estimator
(i.e., the sample mean) is the unbiased estimator for the mean $\mu$
with the lowest variance; the Gauss-Markov theorem establishes this
result in some generality for least-squares estimation, or
equivalently, maximum likelihood estimation under an assumption of
normal noise; see \citep[Section~3.2.2]{HastieTibshiraniFriedman:2009}.











\chapter{Bayesian Data Analysis}\label{bayesian.chapter}

\noindent
\cite{GelmanEtAl:2013} provide the following
characterization of Bayesian data analysis.
%
\begin{quote}
  By Bayesian data analysis, we mean practical methods for making
  inferences from data using probability models for quantities we
  observe and about which we wish to learn.
\end{quote}
%
They go on to describe how Bayesian statistics differs from
frequentist approaches.
%
\begin{quote}
  The essential characteristic of Bayesian methods is their explicit
  use of probability for quantifying uncertainty in inferences based
  on statistical analysis.
\end{quote}
%
Because they view probability as the limit of relative frequencies of
observations, strict frequentists forbid probability statements about
parameters.  Parameters are considered fixed, not random.  

Bayesians also treat parameters as fixed but unknown.  But unlike
frequentists, they make use of both prior distributions over
parameters and posterior distributions over parameters.  These prior
and posterior probabilities and posterior predictive probabilities are
intended to characterize knowledge about the parameters and future
observables.  Posterior distributions form the basis of Bayesian
inference, as described below.

\section{Bayesian Modeling}

\citep{GelmanEtAl:2013} break applied Bayesian modeling
into the following three steps.
%
\begin{enumerate}
\item  Set up a full probability model for all observable and
  unobservable quantities.  This model should be consistent with
  existing knowledge of the data being modeled and how it was
  collected.
\item Calculate the posterior probability of unknown quantities
  conditioned on observed quantities.  The unknowns may include
  unobservable quantities such as parameters and potentially
  observable quantities such as predictions for future observations.
\item Evaluate the model fit to the data.  This includes evaluating
  the implications of the posterior.
\end{enumerate}
%
Typically, this cycle will be repeated until a sufficient fit is
achieved in the third step.  \Stan automates the calculations involved
in the second and third steps.

\section{Bayesian Inference}

\subsection{Basic Quantities}

The mechanics of Bayesian inference follow directly from Bayes's rule.
To fix notation, let $y$ represent observed quantities such as data
and let $\theta$ represent unknown quantities such as parameters and
future observations.  Both $y$ and $\theta$ will be modeled as random.
Let $x$ represent known, but unmodeled quantities such as constants,
hyperparameters, and predictors.

\subsection{Probability Functions}

The probability function $p(y,\theta)$ is the joint probability
function of the data $y$ and parameters $\theta$.  The constants and
predictors $x$ are implicitly understood as being part of the
conditioning.  The conditional probability function $p(y|\theta)$ of
the data $y$ given parameters $\theta$ and constants $x$ is called the
sampling probability function; it is also called the likelihood
function when viewed as a function of $\theta$ for fixed $y$ and $x$.

The probability function $p(\theta)$ over the parameters given the
constants $x$ is called the prior because it characterizes the probability
of the parameters before any data is observed.  The conditional
probability function $p(\theta|y)$ is called the posterior because
it characterizes the probability of parameters given observed data $y$
and constants $x$.  

\subsection{Bayes's Rule}

The technical apparatus of Bayesian inference hinges on the following
chain of equations, known in various forms as Bayes's rule (where
again, the constants $x$ are implicit).
%
\[
\begin{array}{rcll}
p(\theta|y)  & =  & \displaystyle \frac{p(\theta,y)}{p(y)} 
& \mbox{{} \ \ \ \ \ [definition of  conditional probability]}
\\[16pt]
& = & \displaystyle \frac{p(y|\theta) \, p(\theta)}{p(y)}
& \mbox{{} \ \ \ \ \ [chain rule]}
\\[16pt]
& = & \displaystyle \frac{p(y|\theta) \, p(\theta)}
                        {\int_{\Theta} p(y,\theta) \, d\theta}
& \mbox{{} \ \ \ \ \ [law of total probability]}
\\[16pt]
& = & \displaystyle \frac{p(y|\theta) \, p(\theta)}
                        {\int_{\Theta} p(y|\theta) \, p(\theta) \, d\theta}
& \mbox{{} \ \ \ \ \ [chain rule]}
\\[16pt]
& \propto & \displaystyle p(y|\theta) \, p(\theta)
& \mbox{{} \ \ \ \ \ [$y$ is fixed]}
\end{array}
\]
%
Bayes's rule ``inverts'' the probability of the posterior
$p(\theta|y)$, expressing it solely in terms of the likelihood
$p(y|\theta)$ and prior $p(\theta)$ (again, with constants and
predictors $x$ implicit).  The last step is important for \Stan, which
only requires probability functions to be characterized up to a
constant multiplier.

\subsection{Predictive Inference}

The uncertainty in the estimation of parameters $\theta$ from the data
$y$ (given the model) is characterized by the posterior $p(\theta|y)$.
The posterior is thus crucial for Bayesian predictive inference.

If $\tilde{y}$ is taken to represent new, perhaps as yet unknown,
observations, along with corresponding constants and predictors
$\tilde{x}$, then the posterior predictive probability function is
given by
%
\[
p(\tilde{y}|y)
= \int_{\Theta} p(\tilde{y}|\theta) 
                \, p(\theta|y) \, d\theta.
\]
Here, both the original constants and predictors $x$ and the new
constants and predictors $\tilde{x}$ are implicit.  Like the posterior
itself, predictive inference is characterized probabilistically.
Rather than using a point estimate of the parameters $\theta$,
predictions are made based on averaging the predictions over a range
of $\theta$ weighted by the posterior probability $p(\theta|y)$ of
$\theta$ given data $y$ (and constants $x$).

The posterior may also be used to estimate event probabilities.  For
instance, the probability that a parameter $\theta_k$ is greater than
zero is characterized probabilistically by
%
\[
\mbox{Pr}[\theta_k > 0]
= \int_{\Theta} \mbox{I}(\theta_k > 0) \, p(\theta|y) \, d\theta.
\]
%
The indicator function, $\mbox{I}(\phi)$, evaluates to one if the
proposition $\phi$ is true and evaluates to zero otherwise.

Comparisons involving future observables may be carried out in
the same way.  For example, the probability that $\tilde{y}_n >
\tilde{y}_{n'}$ can be characterized using the posterior predictive
probability function as
\[
\mbox{Pr}[\tilde{y}_n > \tilde{y}_{n'}]
= \int_{\Theta} \int_{Y} \mbox{I}(\tilde{y}_n > \tilde{y}_{n'}) \,
p(\tilde{y}|\theta) p(\theta|y) \, d\tilde{y} \, d\theta.
\]


\subsection{Posterior Predictive Checking}

After the parameters are fit to data, they can be used to simulate a
new data set by running the model inferences in the forward
direction.  These replicated data sets can then be compared to the
original data either visually or statistically to assess model fit 
\citep[Chapter 6]{GelmanEtAl:2013}.  

In Stan, posterior simulations can be generated in two ways.  The
first approach is to treat the predicted variables as parameters and
then define their distributions in the model block.  The second
approach, which also works for discrete variables, is to generate
replicated data using random-number generators in the generated
quantities block.



\chapter{Markov Chain Monte Carlo Sampling}\label{mcmc.chapter}

\noindent
Like \BUGS, \Stan uses Markov chain Monte Carlo (\MCMC) techniques to
generate samples from the posterior distribution for inference.  


\section{Monte Carlo Sampling}

Monte Carlo methods were developed to numerically approximate
integrals that are not tractable analytically but for which evaluation
of the function being integrated is tractable
\citep{MetropolisUlam:1949}.

For example, the mean $\mu$ of a probability density $p(\theta)$ is
defined by the integral
\[
\mu = \int_{\Theta} \, \theta \times p(\theta) \, d\theta.
\]
For even a moderately complex Bayesian model, the posterior density
$p(\theta|y)$ leads to an integral that is impossible to evaluate
analytically.  The posterior also depends on the constants and
predictors $x$, but from here, they will just be elided and taken as
given.

Now suppose it is possible to draw independent samples from
$p(\theta)$ and let $\theta^{(1)},\theta^{(2)},\ldots,\theta^{(N)}$ be
$N$ such samples.  A Monte Carlo estimate $\hat{\mu}$ of the mean
$\mu$ of $p(\theta)$ is given by the sample average,
\[
\hat{\mu} = \frac{1}{N} \sum_{n=1}^N \theta^{(n)}.
\]

If the probability function $p(\theta)$ has a finite mean and
variance, the law of large numbers ensures the Monte Carlo estimate
converges to the correct value as the number of samples increases,
\[
\lim_{N \rightarrow \infty} \hat{\mu} = \mu.
\]
Assuming finite mean and variance, estimation error is governed by the
central limit theorem, so that estimation error decreases as the
square root of $N$,
\[
|\mu - \hat{\mu}| \propto \frac{1}{\sqrt{N}}.
\]
Therefore, estimating a mean to an extra decimal place of accuracy
requires one hundred times more samples; adding two decimal places
means ten thousand times as many samples.  This makes Monte Carlo
methods more useful for rough estimates to within a few decimal places
than highly precise estimates.  In practical applications, there is no
point estimating a quantity beyond the uncertainty of the data sample
on which it is based, so this lack of many decimal places of accuracy
is rarely a problem in practice for statistical models.


\section{Markov Chain Monte Carlo Sampling}

Markov chain Monte Carlo (\MCMC) methods were developed for situations
in which it is not straightforward to draw independent samples
\citep{Metropolis:1953}.

A Markov chain is a sequence of random variables $\theta^{(1)},
\theta^{(2)},\ldots$ where each variable is conditionally independent
of all other variables given the value of the previous value.  Thus if
$\theta = \theta^{(1)}, \theta^{(2)},\ldots, \theta^{(N)}$, then
\[
p(\theta) = p(\theta^{(1)}) \prod_{n=2}^N p(\theta^{(n)}|\theta^{(n-1)}).
\]
Stan uses Hamiltonian Monte Carlo to generate a next state in a manner
described in \refchapter{hmc}.

The Markov chains \Stan and other \MCMC samplers generate are ergodic
in the sense required by the Markov chain central limit theorem,
meaning roughly that there is a reasonable chance of reaching
one value of $\theta$ from another.  The Markov chains are also
stationary, meaning that the transition probabilities do not change at
different positions in the chain, so that for $n, n' \geq 0$, the
probability function $p(\theta^{(n+1)}|\theta^{(n)})$ is the same as
$p(\theta^{(n'+1)}|\theta^{(n')})$ (following the convention of
overloading random and bound variables and picking out a probability
function by its arguments).

Stationary Markov chains have an equilibrium distribution on states in
which each has the same marginal probability function, so that
$p(\theta^{(n)})$ is the same probability function as
$p(\theta^{(n+1)})$.  In \Stan, this equilibrium distribution
$p(\theta^{(n)})$ is the probability function $p(\theta)$ being
sampled, typically a Bayesian posterior density.

Using \MCMC methods introduces two difficulties that are not faced by
independent sample Monte Carlo methods.  The first problem is determining
when a randomly initialized Markov chain has converged to its
equilibrium distribution.  The second problem is that the draws from a
Markov chain are correlated, and thus the central limit theorem's
bound on estimation error no longer applies.  These problems are
addressed in the next two sections.


\section{Initialization and Convergence Monitoring}\label{convergence.section}

A Markov chain generates samples from the target distribution only
after it has converged to equilibrium.  Unfortunately, this is only
guaranteed in the limit in theory.  In practice, diagnostics must be
applied to monitor whether the Markov chain(s) have converged.

\subsection{Potential Scale Reduction}

One way to monitor whether a chain has converged to the equilibrium
distribution is to compare its behavior to other randomly initialized
chains.  This is the motivation for the \cite{GelmanRubin:1992}
potential scale reduction statistic, $\hat{R}$.  The $\hat{R}$
statistic measures the ratio of the average variance of samples within
each chain to the variance of the pooled samples across chains; if all
chains are at equilibrium, these will be the same and $\hat{R}$ will
be one.  If the chains have not converged to a common distribution,
the $\hat{R}$ statistic will be greater than one.

Gelman and Rubin's recommendation is that the independent Markov
chains be initialized with diffuse starting values for the parameters
and sampled until all values for $\hat{R}$ are below 1.1.  \Stan
allows users to specify initial values for parameters and it is also
able to draw diffuse random initializations itself.

The $\hat{R}$ statistic is defined for a set of $M$ Markov chains,
$\theta_m$, each of which has $N$ samples $\theta^{(n)}_m$.  The
between-sample variance estimate is
\[
B
= \frac{N}{M-1} \, \sum_{m=1}^M (\bar{\theta}^{(\bullet)}_{m} - \bar{\theta}^{(\bullet)}_{\bullet})^2,
\]
%
where
%
\[
\bar{\theta}_m^{(\bullet)}
= \frac{1}{N} \sum_{n = 1}^N \theta_m^{(n)}
\ \ \ \ \
\mbox{and}
\ \ \ \ \
\bar{\theta}^{(\bullet)}_{\bullet}
= \frac{1}{M} \, \sum_{m=1}^M \bar{\theta}_m^{(\bullet)}.
\]
%
The within-sample variance is
\[
W 
= \frac{1}{M} \, \sum_{m=1}^M s_m^2,
\]
where
\[
s_m^2 = \frac{1}{N-1} \, \sum_{n=1}^N (\theta^{(n)}_m - \bar{\theta}^{(\bullet)}_m)^2.
\]
%
The variance estimator is
\[
\widehat{\mbox{var}}^{+}\!(\theta|y)
= \frac{N-1}{N}\, W \, + \, \frac{1}{N} \, B.
\]
%
Finally, the potential scale reduction statistic is defined by
\[
\hat{R} 
\, = \,
\sqrt{\frac{\widehat{\mbox{var}}^{+}\!(\theta|y)}{W}}.
\]

\subsection{Generalized $\hat{R}$ for Ragged Chains}

Now suppose that each chain may have a different number of samples.
Let $N_m$ be the number of samples in chain $m$.  Now the formula for
the within-chain mean for chain $m$ uses the size of the chain, $N_m$,
\[
\bar{\theta}_m^{(\bullet)}
= \frac{1}{N_m} \sum_{n = 1}^N \theta^{(m)}_n,
\]
as does the within-chain variance estimate,
\[
s_m^2 = \frac{1}{N_m-1} \, \sum_{n=1}^{N_m} (\theta^{(n)}_m - \bar{\theta}^{(\bullet)}_m)^2.
\]
The terms that average over chains, such as
$\bar{\theta}^{(\bullet)}_{\bullet}$, $B$, and $W$, have the same
definition as before to ensure that each chain has the same effect on
the estimate.  If the averages were weighted by size, a single long
chain would dominate the statistics and defeat the purpose of
monitoring convergence with multiple chains.

Because it contains the term $N$, the estimate $\widehat{var}^{+}$
must be generalized.  By expanding the first term,
\[
\frac{N-1}{N}\, W \, 
\ = \ 
\frac{N-1}{N} \frac{1}{M} \, \sum_{m=1}^M
\frac{1}{N-1} \, \sum_{n=1}^N (\theta^{(n)}_m -
\bar{\theta}^{(\bullet)}_m)^2
\ = \
\frac{1}{M} 
\sum_{m=1}^M
\frac{1}{N}
\sum_{n=1}^N (\theta^{(n)}_m -
\bar{\theta}^{(\bullet)}_m)^2,
\]
and the second term,
\[
\frac{1}{N}\, B
\ = \
\frac{1}{M-1} \, \sum_{m=1}^M (\bar{\theta}^{(\bullet)}_{m} - \bar{\theta}^{(\bullet)}_{\bullet})^2.
\]
the variance estimator naturally generalizes to
\[
\widehat{\mbox{var}}^{+}\!(\theta|y)
= 
\frac{1}{M} 
\sum_{m=1}^M
\frac{1}{N_m}
\sum_{n=1}^{N_m} (\theta^{(n)}_m -
\bar{\theta}^{(\bullet)}_m)^2
+
\frac{1}{M-1} \, \sum_{m=1}^M (\bar{\theta}^{(\bullet)}_{m} -
\bar{\theta}^{(\bullet)}_{\bullet})^2.
\]
%
If the chains are all the same length, this definition is equivalent
to the one in the last section.  This generalized variance estimator
and the within-chains variance estimates may be plugged directly into
the formula for $\hat{R}$ from the previous section.


\subsection{Split $\hat{R}$ for Detecting Non-Stationarity}

Before calculating the potential-scale-reduction statistic $\hat{R}$,
each chain may be split into two halves.  This provides an additional
means to detect non-stationarity in the chains.  If one chain involves
gradually increasing values and one involves gradually decreasing
values, they have not mixed well, but they can have $\hat{R}$ values
near unity.  In this case, splitting each chain into two parts leads
to $\hat{R}$ values substantially greater than 1 because the first
half of each chain has not mixed with the second half.  


\subsection{Converence is Global}

A question that often arises is whether it is acceptable to monitor
convergence of only a subset of the parameters or generated
quantities.  The short answer is ``no,'' but this is elaborated
further in this section.

For example, consider the value \code{lp\_\_}, which is the log
posterior density (up to a constant) It is a mistake to declare
convergence in any practical sense if \code{lp\_\_} has not converged,
because different chains are really in different parts of the space.
Yet measuring convergence for \code{lp\_\_} is particularly tricky, as
noted below.

\subsubsection{Asymptotics and transience vs.\ equilibrium}

Markov chain convergence is a global property in the sense that it
does not depend on the choice of function of the parameters that is
monitored.  There is no hard cutoff between pre-convergence
``transience'' and post-convergence ``equilibrium.''  What happens is
that as the number of states in the chain approaches infinity, the
distribution of possible states in the chain approaches the target
distribution and in that limit the expected value of the Monte Carlo
estimator of any integrable function converges to the true
expectation. There is nothing like warmup here, because in the limit,
the effects of initial state are completely washed out.

\subsubsection{Multivariate convergence of functions}

The $\hat{R}$ statistic considers the composition of a Markov chain
and a function, and if the Markov chain has converged then each Markov
chain and function composition will have converged. Multivariate
functions converge when all of their margins have converged by the
Cramer-Wold theorem.  

The transformation from unconstrained space to constrained space is
just another function, so does not effect convergence.

Different functions may have different autocorrelations, but if the
Markov chain has equilibrated then all Markov chain plus function
compositions should be consistent with convergence. Formally, any
function that appears inconsistent is of concern and although it would
be unreasonable to test every function, \code{lp\_\_} and other
measured quantities should at least be consistent.

The obvious difference in \code{lp\_\_} is that it tends to vary
quickly with position and is consequently susceptible to outliers. 

\subsubsection{Finite numbers of states}

The question is what happens for finite numbers of states? If we can
prove a strong geometric ergodicity property (which depends on the
sampler and the target distribution), then one can show that there
exists a finite time after which the chain forgets its initial state
with a large probability. This is both the autocorrelation time and
the warmup time.  But even if you can show it exists and is finite
(which is nigh impossible) you can't compute an actual value
analytically.

So what we do in practice is hope that the finite number of draws is
large enough for the expectations to be reasonably accurate. Removing
warmup iterations improves the accuracy of the expectations but there
is no guarantee that removing any finite number of samples will be
enough.

\subsubsection{Why inconsistent $\hat{R}$?}

There are two things to worry about here.

Firstly, as noted above, for any finite number of draws, there will
always be some residual effect of the initial state, which typically
manifests as some small (or large if the autocorrelation time is huge)
probability of having a large outlier. Functions robust to such
outliers (say, quantiles) will appear more stable and have better
$\hat{R}$. Functions vulnerable to such outliers may show fragility.

Secondly, use of the $\hat{R}$ statistic makes very strong
assumptions. In particular, it assumes that the functions being
considered are Gaussian or it only uses the first two moments and
assumes some kind of independence.  The point is that strong
assumptions are made that do not always hold. In particular, the
distribution for the log posterior density (\code{lp\_\_}) almost
never looks Gaussian, instead it features long tails that can lead to
large $\hat{R}$ even in the large $N$ limit.  Tweaks to $\hat{R}$,
such as using quantiles in place of raw values, have the flavor of
making the samples of interest more Gaussian and hence the $\hat{R}$
statistic more accurate.

\subsubsection{Final words on convergence monitoring}

``Convergence'' is a global property and holds for all integrable
functions at once, but employing the $\hat{R}$ statistic requires
additional assumptions and thus may not work for all functions equally
well.

Note that if you just compare the expectations between chains then we
can rely on the Markov chain asymptotics for Gaussian distributions
and can apply the standard tests.



\section{Effective Sample Size}\label{effective-sample-size.section}

The second technical difficulty posed by \MCMC methods is that the
samples will typically be autocorrelated within a chain.  This
increases the uncertainty of the estimation of posterior quantities of
interest, such as means, variances or quantiles.  

A nice introductory reference for analyzing MCMC results in general
and effective sample size in particular is \citep{Geyer:2011}.  The
particular calculations used by Stan follow those for split-$\hat{R}$,
which involve both cross-chain (mean) and within-chain calculations
(autocorrelation); they were introduced in this manual and explained
in more detail in \citep{GelmanEtAl:2013}.

\subsection{Definition of Effective Sample Size}

The amount by which autocorrelation within the chains increases
uncertainty in estimates can be measured by effective sample size
({\sc ess}).  Given independent samples, the central limit theorem
bounds uncertainty in estimates based on the number of samples $N$.
Given dependent samples, the number of independent samples is replaced
with the effective sample size $N_{\mbox{\scriptsize eff}}$, which is
the number of independent samples with the same estimation power as
the $N$ autocorrelated samples.  For example, estimation error is
proportional to $1/\sqrt{N_{\mbox{\scriptsize eff}}}$ rather than
$1/\sqrt{N}$.

The effective sample size of a sequence is defined in terms of the
autocorrelations within the sequence at different lags.  The
autocorrelation $\rho_t$ at lag $t \geq 0$ for a chain with joint
probability function $p(\theta)$ with mean $\mu$ and variance
$\sigma^2$ is defined to be
\[
\rho_t 
= 
\frac{1}{\sigma^2} \, \int_{\Theta} (\theta^{(n)} - \mu)
(\theta^{(n+t)} - \mu) \, p(\theta) \, d\theta.
\]
This is just the correlation between the two chains offset by $t$
positions.  Because we know $\theta^{(n)}$ and $\theta^{(n+t)}$ have
the same marginal distribution in an \MCMC setting, multiplying the
two difference terms and reducing yields
\[
\rho_t
= 
\frac{1}{\sigma^2} \, \int_{\Theta} \theta^{(n)} \, \theta^{(n+t)} \, p(\theta) \, d\theta.
\]

The effective sample size of $N$ samples generated by a process with
autocorrelations $\rho_t$ is defined by
\[
N_{\mbox{\scriptsize eff}}
\ = \
\frac{N}{\sum_{t = -\infty}^{\infty} \rho_t}
\ = \
\frac{N}{1 + 2 \sum_{t = 1}^{\infty} \rho_t}.
\]

\subsection{Estimation of Effective Sample Size}

In practice, the probability function in question cannot be tractably
integrated and thus the autocorrelation cannot be calculated, nor the
effective sample size.  Instead, these quantities must be estimated
from the samples themselves.  The rest of this section describes a
variogram-based estimator for autocorrelations, and hence effective sample
size, based on multiple chains. For simplicity, each chain
$\theta_m$ will be assumed to be of length $N$.

One way to estimate the effective sample size is based on the
variograms $V_t$ at lag $t \in \setlist{0,1\ldots}$.  The variograms are
defined as follows for (univariate) samples $\theta_m^{(n)}$, where $m \in
\setlist{1,\ldots,M}$ is the chain, and $N_m$ is the number of samples
in chain $m$.
\[
V_t = 
\frac{1}{M}
\,
\sum_{m=1}^M 
\
\left(
\frac{1}{N_m - t}
\sum_{n=t+1}^{N_m}
\left(
\theta_m^{(n)} - \theta_m^{(n-t)}
\right)^2
\right).
\]
%
The variogram along with the multi-chain variance estimate
$\widehat{\mbox{var}}^{+}$ introduced in the previous section can be
used to estimate the autocorrelation at lag $t$ as
\[
\hat{\rho}_t
= 1 - \frac{\displaystyle V_t}{
            \displaystyle 2 \, \widehat{\mbox{var}}^{+}}.
\]
If the chains have not converged, the variance estimator
$\widehat{\mbox{var}}^{+}$ will overestimate variance, 
leading to an overestimate of autocorrelation and an underestimate
effective sample size.

Because of the noise in the correlation estimates $\hat{\rho}_t$ as $t$
increases, typically only the initial estimates of $\hat{\rho}_t$
where $\hat{\rho}_t > 0$ will be used.  Setting $T'$ to be the
first lag such that $\rho_{T' + 1} < 0$, 
%
\[
T' = \arg\min_t \ \hat{\rho}_{t+1} < 0,
\]
the effective sample size estimator is defined as
\[
\hat{N}_{\mbox{\scriptsize eff}}
= 
\frac{MN}
     {1 + \sum_{t=1}^{T'} \hat{\rho}_t}.
\]
%
Exact autocorrelations can happen only on odd lags \citep{Geyer:2011}.
By summing over pairs, the paired autocorrelation is guaranteed to be
positive modulo estimator noise.  This is the motivation behind the
many termination criterion of \cite{Geyer:2011}. Stan does not (yet)
do the paired expectations because NUTS almost by construction avoids
the negative autocorrelation regime.  Thus terminating at the first
negative autocorrelation is a reasonable approximation for stopping
when the noise in the autocorrelation estimator dominates.

Stan carries out the autocorrelation computations for all lags
simultaneously using Eigen's fast Fourier transform (FFT) package with
appropriate padding; see \citep{Geyer:2011} for more detail on using
FFT for autocorrelation calculations.


\subsection{Thinning Samples}

In the typical situation, the autocorrelation, $\rho_t$, decreases as
the lag, $t$, increases.  When this happens, thinning the samples will
reduce the autocorrelation.  For instance, consider generating one
thousand samples in one of the following two ways.
%
\begin{enumerate}
\item Generate 1000 samples after convergence and save all of
  them.
\item Generate 10,000 samples after convergence and save every tenth
  sample.
\end{enumerate}
%
Even though both produce one thousand samples, the second approach
with thinning will produce more effective samples.  That's because the
autocorrelation $\rho_t$ for the thinned sequence is equivalent to
$\rho_{10t}$ in the unthinned samples, so the sum of the autocorrelations
will be lower and thus the effective sample size higher.  

On the other hand, if memory and data storage are no object, saving all
ten thousand samples will have a higher effective sample size than
thinning to one thousand samples.



\chapter{Transformations of Constrained Variables}\label{variable-transforms.chapter}

\noindent
To avoid having to deal with constraints while simulating the
Hamiltonian dynamics during sampling, every (multivariate) parameter
in a \Stan model is transformed to an unconstrained variable behind
the scenes by the model compiler.  The transform is based on the
constraints, if any, in the parameter's definition.  Scalars or the
scalar values in vectors, row vectors or matrices may be constrained
with lower and/or upper bounds.  Vectors may alternatively be
constrained to be ordered, positive ordered, or simplexes.  Matrices
may be constrained to be correlation matrices or covariance matrices.
This chapter provides a definition of the transforms used for each
type of variable.

\Stan converts models to \Cpp classes which define probability
functions with support on all of $\reals^K$, where $K$ is the number
of unconstrained parameters needed to define the constrained
parameters defined in the program.  The \Cpp classes also include
code to transform the parameters from unconstrained to constrained and
apply the appropriate Jacobians.


\section{Changes of Variables}\label{change-of-variables.section}

The support of a random variable $X$ with density $p_X(x)$ is that
subset of values for which it has non-zero density,
%
\[
\mbox{supp}(X) = \{ x | p_X(x) > 0 \}.
\]

If $f$ is a total function defined on the support of $X$, then $Y =
f(X)$ is a new random variable.  This section shows how to compute the
probability density function of $Y$ for well-behaved transforms $f$.
The rest of the chapter details the transforms used by \Stan.



\subsection{Univariate Changes of Variables}

Suppose $X$ is one dimensional and $f: \mbox{supp}(X) \rightarrow
\reals$ is a one-to-one, monotonic function with a differentiable
inverse $f^{-1}$.  Then the density of $Y$ is given by
%
\[
p_Y(y) = p_X(f^{-1}(y))  
         \,
         \left| \, \frac{d}{dy} f^{-1}(y)\, \right|.
\]
The absolute derivative of the inverse transform measures how the
scale of the transformed variable changes with respect to the
underlying variable.


\subsection{Multivariate Changes of Variables}

The multivariate generalization of an absolute derivative is a
Jacobian, or more fully the absolute value of the determinant of the
Jacobian matrix of the transform.  The Jacobian matrix measures the change of
each output variable relative to every input variable and the absolute
determinant uses that to determine the differential change in volume
at a given point in the parameter space.

Suppose $X$ is a $K$-dimensional random variable with probability
density function $p_X(x)$.  A new random variable $Y = f(X)$ may be
defined by transforming $X$ with a suitably well-behaved function $f$.
It suffices for what follows to note that if $f$ is one-to-one
and its inverse $f^{-1}$ has a well-defined Jacobian, then the
density of $Y$ is
%
\[
p_Y(y) = p_X(f^{-1}(y)) \, \left| \, \det \, J_{f^{-1}}(y) \, \right|,
\]
%
where $\det{}$ is the matrix determinant operation and $J_{f^{-1}}(y)$
is the Jacobian matrix of $f^{-1}$ evaluated at $y$.  Taking $x =
f^{-1}(y)$, the Jacobian matrix is defined by
\[
J_{f^{-1}}(y) = 
\left[
\begin{array}{ccc}\displaystyle
\frac{\partial x_1}{\partial y_1}
& \cdots
& \displaystyle \frac{\partial x_1}{\partial y_{K}}
\\[6pt]
\vdots & \vdots & \vdots
\\
\displaystyle\frac{\partial x_{K}}{\partial y_1}
& \cdots
& \displaystyle\frac{\partial x_{K}}{\partial y_{K}}
\end{array}
\right].
\]
%
If the Jacobian matrix is triangular, the determinant reduces to the
product of the diagonal entries,
%
\[
\det \, J_{f^{-1}}(y)
= \prod_{k=1}^K \frac{\partial x_k}{\partial y_k}.
\]
%
Triangular matrices naturally arise in situations where the variables
are ordered, for instance by dimension, and each variable's
transformed value depends on the previous variable's transformed
values.  Diagonal matrices, a simple form of triangular matrix, arise
if each transformed variable only depends on a single untransformed
variable.

\section{Lower Bounded Scalar}\label{lower-bound-transform.section}

\Stan uses a logarithmic transform for lower and upper bounds.  

\subsection{Lower Bound Transform}

If a variable $X$ is declared to have lower bound $a$, it is
transformed to an unbounded variable $Y$, where
%
\[
Y = \log(X - a).
\]

\subsection{Lower Bound Inverse Transform}
%
The inverse of the lower-bound transform maps an unbounded
variable $Y$ to a variable $X$ that is bounded below by $a$ by
%
\[
X = \exp(Y) + a.
\]

\subsection{Absolute Derivative of the Lower Bound Inverse Transform}

The absolute derivative of the inverse transform is
\[
\left| \,
\frac{d}{dy} \left( \exp(y) + a \right)
\, \right|
= \exp(y).
\]
Therefore, given the density $p_X$ of $X$, the density of $Y$ is 
%
\[
p_Y(y) 
= p_X\!\left( \exp(y) + a \right) \cdot \exp(y).
\]


\section{Upper Bounded Scalar}

\Stan uses a negated logarithmic transform for upper bounds.

\subsection{Upper Bound Transform}

If a variable $X$ is declared to have an upper bound $b$, it is
transformed to the unbounded variable $Y$ by
%
\[
Y = \log(b - X).
\]

\subsection{Upper Bound Inverse Transform}
%
The inverse of the upper bound transform converts the unbounded
variable $Y$ to the variable $X$ bounded above by $b$ through
%
\[
X = b - \exp(Y).
\]

\subsection{Absolute Derivative of the Upper Bound Inverse Transform}

The absolute derivative of the inverse of the upper bound transform is 
\[
\left| \,
\frac{d}{dy} \left( b - \exp(y) \right)
\, \right|
= \exp(y).
\]
%
Therefore, the density of the unconstrained variable $Y$ is defined in
terms of the density of the variable $X$ with an upper bound of $b$ by
%
\[
p_Y(y) 
 =   p_X \!\left( b - \exp(y) \right) \cdot \exp(y).
\]


\section{Lower and Upper Bounded Scalar}\label{logit-transform-jacobian.section}

For lower and upper-bounded variables, \Stan uses a scaled and
translated log-odds transform.

\subsection{Log Odds and the Logistic Sigmoid}

The log-odds function is defined for $u \in (0,1)$ by
%
\[
\mbox{logit}(u) = \log \frac{u}{1 - u}.
\]
% 
The inverse of the log odds function is the logistic sigmoid, defined 
for $v \in (-\infty,\infty)$ by
%
\[
\mbox{logit}^{-1}(v) = \frac{1}{1 + \exp(-v)}.
\]
% 
The derivative of the logistic sigmoid is
%
\[
\frac{d}{dy} \mbox{logit}^{-1}(y) 
= \mbox{logit}^{-1}(y) \cdot \left( 1 - \mbox{logit}^{-1}(y) \right).
\]

\subsection{Lower and Upper Bounds Transform}

For variables constrained to be in the open interval $(a,b)$, \Stan
uses a scaled and translated log-odds transform.  If variable $X$ is
declared to have lower bound $a$ and upper bound $b$, then it is
transformed to a new variable $Y$, where
%
\[
Y = \mbox{logit} \left( \frac{X - a}{b - a} \right).
\]
%

\subsection{Lower and Upper Bounds Inverse Transform}

The inverse of this transform is
%
\[
X = a + (b - a) \cdot \mbox{logit}^{-1}(Y).
\]
%

\subsection{Absolute Derivative of the Lower and Upper Bounds Inverse
  Transform}

The absolute derivative of the inverse transform is given by
\[
\left|  
  \frac{d}{dy} 
    \left( 
      a + (b - a) \cdot \mbox{logit}^{-1}(y)
    \right)
  \right|
= (b - a)
    \cdot \mbox{logit}^{-1}(y)
    \cdot \left( 1 - \mbox{logit}^{-1}(y) \right).
\]
Therefore, the density of the transformed variable $Y$ is
%
\[
p_Y(y) 
= 
 p_X \! \left( a + (b - a) \cdot \mbox{logit}^{-1}(y) \right)
    \cdot (b - a)
    \cdot \mbox{logit}^{-1}(y)
    \cdot \left( 1 - \mbox{logit}^{-1}(y) \right).
\]
%
Despite the apparent complexity of this expression, most of the terms
are repeated and thus only need to be evaluated once.  Most
importantly, $\mbox{logit}^{-1}(y)$ only needs to be evaluated once,
so there is only one call to $\exp(-y)$.


\section{Ordered Vector}

For some modeling tasks, a vector-valued random variable $X$ is
required with support on ordered sequences.  One example is the set of
cut points in ordered logistic regression (see \refsection{ordered-logistic}).

In constraint terms, an ordered $K$-vector $x \in \reals^K$ satisfies
\[
x_k < x_{k+1}
\]
%
for $k \in \setlist{1,\ldots,K-1}$.


\subsection{Ordered Transform}

\Stan's transform follows the constraint directly.  It maps an
increasing vector $x \in \reals^{K}$ to an unconstrained vector $y \in
\reals^K$ by setting
%
\[
y_k
= 
\left\{
\begin{array}{ll}
x_1 & \mbox{if } k = 1, \mbox{ and}
\\[4pt]
\log \left( x_{k} - x_{k-1} \right) & \mbox{if } 1 < k \leq K.
\end{array}
\right.
\] 

\subsection{Ordered Inverse Transform}

The inverse transform for an unconstrained $y \in \reals^K$ to an
ordered sequence $x \in \reals^K$ is defined by the recursion
%
\[
x_k
= 
\left\{
\begin{array}{ll} 
y_1 & \mbox{if } k = 1, \mbox{ and}
\\[4pt]
x_{k-1} + \exp(y_k) & \mbox{if } 1 < k \leq K.
\end{array}
\right.
\]
%
$x_k$ can also be expressed iteratively as
\[
x_k = y_1 + \sum_{k'=2}^k \exp(y_{k'}).
\]

\subsection{Absolute Jacobian Determinant of the Ordered
  Inverse Transform}

The Jacobian of the inverse transform $f^{-1}$ is lower triangular,
with diagonal elements for $1 \leq k \leq K$ of
\[
J_{k,k} = 
\left\{
\begin{array}{ll} 
1 & \mbox{if } k = 1, \mbox{ and}
\\[4pt]
\exp(y_k) & \mbox{if } 1 < k \leq K.
\end{array}
\right.
\]
%
Because $J$ is triangular, the absolute Jacobian determinant is
%
\[
\left| \, \det \, J \, \right|
\ = \ 
\left| \, \prod_{k=1}^K J_{k,k} \, \right|
\ = \ 
\prod_{k=2}^K \exp(y_k).
\]

Putting this all together, if $p_X$ is the density of $X$, then the
transformed variable $Y$ has density $p_Y$ given by
%
\[
p_Y(y)
= p_X(f^{-1}(y)) 
\
\prod_{k=2}^K \exp(y_k).
\]


\section{Unit Simplex}\label{simplex-transform.section}

Variables constrained to the unit simplex show up in multivariate
discrete models as both parameters (categorical and multinomial) and
as variates generated by their priors (Dirichlet and multivariate
logistic).

The unit $K$-simplex is the set of points $x \in \reals^K$ such that
for $1 \leq k \leq K$, 
\[ 
x_k > 0,
\] 
and
\[
\sum_{k=1}^K x_k = 1.
\]
%   
An alternative definition is to take the convex closure of the
vertices.  For instance, in 2-dimensions, the simplex vertices are the
extreme values $(0,1)$, and $(1,0)$ and the unit 2-simplex is the line
connecting these two points; values such as $(0.3,0.7)$ and
$(0.99,0.01)$ lie on the line.  In 3-dimensions, the basis is
$(0,0,1)$, $(0,1,0)$ and $(1,0,0)$ and the unit 3-simplex is the
boundary and interior of the triangle with these vertices.  Points in
the 3-simplex include $(0.5,0.5,0)$, $(0.2,0.7,0.1)$ and all other
triplets of non-negative values summing to 1.  

As these examples illustrate, the simplex always picks out a subspace
of $K-1$ dimensions from $\reals^K$.  Therefore a point $x$ in the
$K$-simplex is fully determined by its first $K-1$ elements $x_1, x_2,
\ldots, x_{K-1}$, with
%
\[
x_K = 1 - \sum_{k=1}^{K-1} x_k.
\]
%

\subsection{Unit Simplex Inverse Transform}

Stan's unit simplex inverse transform may be understood using the
following stick-breaking metaphor.%
%
\footnote{For an alternative derivation of the same transform using
  hyperspherical coordinates, see \citep{Betancourt:2010}.}
%
\begin{quote}
  Take a stick of unit length (i.e., length 1).  Break a piece off and
  label it as $x_1$, and set it aside.  Next, break a piece off what's
  left, label it $x_2$, and set it aside.  Continue doing this until
  you have broken off $K-1$ pieces labeled $(x_1,\ldots,x_{K-1})$.
  Label what's left of the original stick $x_K$.  The vector $x =
  (x_1,\ldots,x_{K})$ is obviously a unit simplex because each piece
  has non-negative length and the sum of their lengths is 1.
\end{quote}
%
This full inverse mapping requires the breaks to be represented as the
fraction in $(0,1)$ of the original stick that is broken off.  These
break ratios are themselves derived from unconstrained values in
$(-\infty,\infty)$ using the inverse logit transform as described
above for unidimensional variables with lower and upper bounds.

More formally, an intermediate vector $z \in \reals^{K-1}$, whose
coordinates $z_k$ represent the proportion of the stick broken off in
step $k$, is defined elementwise for $1 \leq k < K$ by
%
\[
z_k = \mbox{logit}^{-1} \left( y_k 
                             + \log \left( \frac{1}{K - k}
                                            \right)
                       \right).
\]
%
The logit term
$\log\left(\frac{1}{K-k}\right) (i.e., \mbox{logit}\left(\frac{1}{K-k+1}\right)$) in
the above definition adjusts the transform so that a
zero vector $y$ is mapped to the simplex $x = (1/K,\ldots,1/K)$.  For instance, if
$y_1 = 0$, then $z_1 = 1/K$; if $y_2 = 0$, then $z_2 = 1/(K-1)$; and
if $y_{K-1} = 0$, then $z_{K-1} = 1/2$.  

The break proportions $z$ are applied to determine the stick sizes and
resulting value of $x_k$ for $1 \leq k < K$ by
%
\[
x_k = 
\left( 1 - \sum_{k'=1}^{k-1} x_{k'} \right) z_k.
\]
%
The summation term represents the length of the original stick left at
stage $k$.  This is multiplied by the break proportion $z_k$ to yield
$x_k$.  Only $K-1$ unconstrained parameters are required, with
the last dimension's value $x_K$ set to the length of the remaining
piece of the original stick,
\[
x_K = 1 - \sum_{k=1}^{K-1} x_k.
\]

\subsection{Absolute Jacobian Determinant of the Unit-Simplex
  Inverse Transform}

The Jacobian $J$ of the inverse transform $f^{-1}$ is
lower-triangular, with diagonal entries
\[
J_{k,k}
=
\frac{\partial x_k}{\partial y_k}
=
\frac{\partial x_k}{\partial z_k} \,
\frac{\partial z_k}{\partial y_k},
\]
%
where
\[
\frac{\partial z_k}{\partial y_k} 
= \frac{\partial}{\partial y_k} 
   \mbox{logit}^{-1} \left(
                       y_k + \log \left( \frac{1}{K-k}
                                          \right)
                    \right)
= z_k (1 - z_k),
\]
%
and
%
\[
\frac{\partial x_k}{\partial z_k}
=
\left( 
  1 - \sum_{k' = 1}^{k-1} x_{k'}
   \right)
.
\]
%
This definition is recursive, defining $x_k$ in terms of
$x_{1},\ldots,x_{k-1}$.

Because the Jacobian $J$ of $f^{-1}$ is lower triangular and positive, its
absolute determinant reduces to
%
\[
\left| \, \det J \, \right|
\ = \
\prod_{k=1}^{K-1} J_{k,k}
\ = \
\prod_{k=1}^{K-1} 
z_k
\, 
(1 - z_k)
\
\left(
1 - \sum_{k'=1}^{k-1} x_{k'}
\right)
.
\]
%
Thus the transformed variable $Y = f(X)$ has a density given by
%
\[
p_Y(y) 
= p_X(f^{-1}(y))
\,
\prod_{k=1}^{K-1} 
z_k
\, 
(1 - z_k)
\
\left(
1 - \sum_{k'=1}^{k-1} x_{k'}
\right)
.
\]
%
Even though it is expressed in terms of intermediate values $z_k$,
this expression still looks more complex than it is. The exponential
function need only be evaluated once for each unconstrained parameter
$y_k$; everything else is just basic arithmetic that can be computed
incrementally along with the transform.

\subsection{Unit Simplex Transform}

The transform $Y = f(X)$ can be derived by reversing the stages of the
inverse transform.  Working backwards, given the break proportions
$z$, $y$ is defined elementwise by
%
\[
y_k 
= \mbox{logit}(z_k)
- \mbox{log}\left(
   \frac{1}{K-k}
   \right)
.
\]
%
The break proportions $z_k$ are defined to be the ratio of $x_k$ to
the length of stick left after the first $k-1$ pieces have been broken
off, 
%
\[
z_k 
= \frac{x_k}
       {1 - \sum_{k' = 1}^{k-1} x_{k'}}
.
\]

\section{Unit Vector}
Unit vectors show up in directional statistics.

The $n$-sphere is the set of points $x \in \reals^n$ such that 
\[
\Vert x \Vert^2 = \sum_{i=1}^n x_i^2 = 1\ .
\]

\subsection{Unit Vector Inverse Transform}

To parametrize unit length vectors, we use hyperspherical coordinates.
The unconstrained vector $y \in \reals^{n-1}$ is a set of angles which
relates to the unit vector as follows:
\begin{eqnarray*}
x_1 & = & \cos(y_1) \\
x_2 & = & \cos(y_2) \sin(y_1) \\
x_3 & = & \cos(y_3) \sin(y_1) \sin(y_2) \\
& \vdots & \\
x_i & = & \cos(y_i) \prod_{j=1}^{i-1} \sin(y_j) \\
& \vdots & \\
x_n & = & \prod_{j=1}^{n-1} \sin(y_j).
\end{eqnarray*}
Note that, in practice, we use $y_i = \hat{y}_i + \frac{\pi}{2}$ and use $\hat{y}_i$ as the unconstrained parameters to avoid a singularity at
$y_i = 0$.


\subsection{Absolute Jacobian Determinant of the Unit Vector
  Inverse Transform}
To derive the determinant of the Jacobian we first add a radius coordinate
$r=1$ which multiplies the vector $x$.
The Jacobian, $J$, can be shown to be lower triangular, so its
determinant is just the product of diagonal entries and the 
absolute value of the determinant is
\[
\left| \, \det J \, \right| = \left|r^{n-1} \sin^{n-2}(y_1) \sin^{n-3}(y_2) ... \sin(y_{n-2})\right| .
\]


\section{Correlation Matrices}\label{correlation-matrix-transform.section}

A $K \times K$ correlation matrix $x$ must be is a symmetric, so that
%
\[
x_{k,k'} = x_{k',k}
\]
for all $k,k' \in \setlist{1,\ldots,K}$, it must have a unit diagonal,
so that 
\[
x_{k,k} = 1
\]
for all $k \in \setlist{1,\ldots,K}$, and it must be positive
definite, so that for every non-zero $K$-vector $a$,
\[
a^{\top} x a > 0.
\]
To deal with this rather complicated constraint, \Stan implements the
transform of \cite{LewandowskiKurowickaJoe:2009}.  The number of free
parameters required to specify a $K \times K$ correlation matrix is $\binom{K}{2}$.

\subsection{Correlation Matrix Inverse Transform}

It is easiest to specify the inverse, going from its $\binom{K}{2}$
parameter basis to a correlation matrix.  The basis will actually be
broken down into two steps.  To start, suppose $y$ is a vector
containing $\binom{K}{2}$ unconstrained values.  These are first
transformed via the bijective function $\tanh : \reals \rightarrow
(0,1)$
%
\[
\tanh x = \frac{\exp(2x) - 1}{\exp(2x) + 1}.
\]
%
Then, define a $K \times K$ matrix $z$, the upper triangular values of
which are filled by row with the transformed values.  For example, in
the $4 \times 4$ case, there are $\binom{4}{2}$ values arranged as
%
\[
z 
=
\left[
\begin{array}{cccc}
0 & \tanh y_1 & \tanh y_2 & \tanh y_4
\\
0 & 0 & \tanh y_3 & \tanh y_5
\\
0 & 0 & 0 & \tanh y_6
\\
0 & 0 & 0 & 0
\end{array}
\right]
.
\]
%
Lewandowski et al.\ show how to bijectively map the array $z$ to a correlation
matrix $x$.  The entry $z_{i,j}$ for $i < j$ is interpreted as the
canonical partial correlation (\CPC) between $i$ and $j$, which is the
correlation between $i$'s residuals and $j$'s residuals when both $i$
and $j$ are regressed on all variables $i'$ such that $i'< i$.
In the case of $i=1$, there are no earlier variables, 
so $z_{1,j}$ is just the Pearson correlation between $i$ and $j$.

In \Stan, the \LKJ transform is reformulated in terms of a Cholesky factor $w$
of the final correlation matrix, defined for $1 \leq i,j \leq K$ by
%
\[
w_{i,j} = 
\left\{
\begin{array}{cl}
%
0 & \mbox{if } i > j,
\\[4pt]
1 & \mbox{if } 1 = i = j,
\\[12pt]
\prod_{i'=1}^{i - 1} \left( 1 - z_{i'\!,\,j}^2 \right)^{1/2}
& \mbox{if } 1 < i = j,
\\[12pt]
z_{i,j} & \mbox{if } 1 = i < j, \mbox{ and}
\\[12pt]
z_{i,j} \, \prod_{i'=1}^{i-1} \left( 1 - z_{i'\!,\,j}^2 \right)^{1/2}
& \mbox{ if } 1 < i < j.
%
\end{array}
\right.
\]
%
This does not require as much computation per matrix entry as it may appear; 
calculating the rows in terms of earlier rows yields the more
manageable expression 
%
\[
w_{i,j} = 
\left\{
\begin{array}{cl}
%
0 & \mbox{if } i > j,
\\[4pt]
1 & \mbox{if } 1 = i = j, 
\\[8pt]
z_{i,j} & \mbox{if } 1 = i < j, \mbox{ and}
\\[8pt]
z_{i,j} \ w_{i-1,j} \left( 1 - z_{i-1,j}^2 \right)^{1/2}
& \mbox{ if } 1 < i \leq j.
%
\end{array}
\right.
\]
Given the upper-triangular Cholesky factor $w$, the final correlation
matrix is
\[
x = w^{\top} w.
\]

Lewandowski et al.\ show that the determinant of the correlation
matrix can be defined in terms of the canonical partial correlations
as
%
\[
\mbox{det} \, x = \prod_{i=1}^{K-1} \ \prod_{j=i+1}^K \ (1 - z_{i,j}^2)
 = \prod_{1 \leq i < j \leq K} (1 - z_{i,j}^2),
\]

\subsection{Absolute Jacobian Determinant of the Correlation
  Matrix Inverse Transform}
From the inverse of equation 11 in \cite{LewandowskiKurowickaJoe:2009},
the absolute Jacobian determinant is
%
\[
\sqrt{\prod_{i=1}^{K-1}\prod_{j=i+1}^K \left(1-z_{i,j}^2\right)^{K-i-1}} \
\times \prod_{i=1}^{K-1}\prod_{j=i+1}^K 
\frac{\partial \tanh z_{i,j}}{\partial y_{i,j}}
\]
\subsection{Correlation Matrix Transform}

The correlation transform is defined by reversing the steps of the
inverse transform defined in the previous section.  

Starting with a correlation matrix $x$, the first step is to find the
unique upper triangular $w$ such that $x = w w^{\top}$.  Because $x$
is positive definite, this can be done by applying the Cholesky
decomposition,
\[
w = \mbox{chol}(x).
\]


The next step from the Cholesky factor $w$ back to the array $z$ of
{\CPC}s is simplified by the ordering of the elements in the
definition of $w$, which when inverted yields
%
\[
z_{i,j} =
\left\{
\begin{array}{cl}
0 & \mbox{if } i \leq j,
\\[8pt]
w_{i,j} & \mbox{if } 1 = i < j, \mbox{ and}
\\[8pt]
{w_{i,j}}
\
\prod_{i'=1}^{i-1} \left( 1 - z_{i'\!,j}^2 \right)^{-2}
& \mbox{if } 1 < i < j.
\end{array}
\right.
\]
The final stage of the transform reverses the hyperbolic tangent
transform, which is defined by 
\[
\tanh^{-1} v = \frac{1}{2} \log \left( \frac{1 + v}{1 - v} \right).
\]
The inverse hyperbolic tangent function, $\tanh^{-1}$, is also called
the Fisher transformation.


\section{Covariance Matrices}

A $K \times K$ matrix is a covariance matrix if it is symmetric and
positive definite (see the previous section for definitions).  It
requires $K + \binom{K}{2}$ free parameters to specify a $K \times K$
covariance matrix.


\subsection{Covariance Matrix Transform}

\Stan's covariance transform is based on a Cholesky decomposition
composed with a log transform of the positive-constrained diagonal
elements.%
%
\footnote{An alternative to the transform in this section, which can
  be coded directly in \Stan, is to parameterize a covariance matrix
  as a scaled correlation matrix.  An arbitrary $K \times K$
  covariance matrix $\Sigma$ can be expressed in terms of a $K$-vector
  $\sigma$ and correlation matrix $\Omega$ as
  \[
  \Sigma = \mbox{diag}(\sigma) \times \Omega \times \mbox{diag}(\sigma),
  \]
  so that each entry is just a deviation-scaled correlation,
  \[
  \Sigma_{m,n} = \sigma_m \times \sigma_n \times \Omega_{m,n}.
  \]
}

If $x$ is a covariance matrix (i.e., a symmetric, positive definite
matrix), then there is a unique lower-triangular matrix $z =
\mbox{chol}(x)$ with positive diagonal entries, called a Cholesky
factor, such that
\[
x = z \, z^{\top}.
\]
The off-diagonal entries of the Cholesky factor $z$ are unconstrained,
but the diagonal entries $z_{k,k}$ must be positive for $1 \leq k
\leq K$.

To complete the transform, the diagonal is log-transformed to produce
a fully unconstrained lower-triangular matrix $y$ defined by
\[
y_{m,n} = 
\left\{
\begin{array}{cl}
0 & \mbox{if } m < n,
\\[4pt]
\log z_{m,m} & \mbox{if } m = n, \mbox{ and}
\\[4pt]
z_{m,n} & \mbox{if } m > n.
\end{array}
\right.
\]

\subsection{Covariance Matrix Inverse Transform}

The inverse transform reverses the two steps of the transform.
Given an unconstrained lower-triangular $K \times K$ matrix $y$, the
first step is to recover the intermediate matrix $z$ by reversing the
log transform,
\[
z_{m,n} = 
\left\{
\begin{array}{cl}
0 & \mbox{if } m < n,
\\[4pt]
\exp(y_{m,m}) & \mbox{if } m = n, \mbox{ and}
\\[4pt]
y_{m,n} & \mbox{if } m > n.
\end{array}
\right.
\]
%
The covariance matrix $x$ is recovered from its Cholesky factor $z$ by
taking
%
\[
x = z \, z^{\top}.
\]

\subsection{Absolute Jacobian Determinant of the Covariance
  Matrix Inverse Transform}

The Jacobian is the product of the Jacobians of the exponential
transform from the unconstrained lower-triangular matrix $y$ to matrix
$z$ with positive diagonals and the product transform from the
Cholesky factor $z$ to $x$.

The transform from unconstrained $y$ to Cholesky factor $z$ has a
diagonal Jacobian matrix, the absolute determinant of which is thus
%
\[
\prod_{k=1}^K  \frac{\partial}{\partial_{y_{k,k}}} \, \exp(y_{k,k})
\ = \ 
\prod_{k=1}^K \exp(y_{k,k})
\ = \
\prod_{k=1}^K z_{k,k}.
\]

The Jacobian matrix of the second transform from the Cholesky factor $z$ to
the covariance matrix $x$ is also triangular, with diagonal entries
corresponding to pairs $(m,n)$ with $m \geq n$, defined by
\[
\frac{\partial}{\partial z_{m,n}}
\left( z \, z^{\top} \right)_{m,n}
\ = \
\frac{\partial}{\partial z_{m,n}}
\left( \sum_{k=1}^K z_{m,k} \, z_{n,k} \right)
\ = \
\left\{
\begin{array}{cl}
2 \, z_{n,n} & \mbox{if } m = n \mbox{ and }
\\[4pt]
z_{n,n} & \mbox{if } m > n.
\end{array}
\right.
\]
%
The absolute Jacobian determinant of the second transform is thus
\[
2^{K} \ \prod_{m = 1}^{K} \ \prod_{n=1}^{m} z_{n,n}
\ = \
\prod_{n=1}^K \ \prod_{m=n}^K z_{n,n}
\ = \ 
2^{K} \ \prod_{k=1}^K z_{k,k}^{K - k + 1}.
\]
Finally, the full absolute Jacobian determinant of the inverse
of the covariance matrix transform from the unconstrained lower-triangular 
$y$ to a symmetric, positive definite matrix $x$ is the product of the
Jacobian determinants of the exponentiation and product transforms,
\[
\left( \prod_{k=1}^K z_{k,k} \right)
\left( 
2^{K} \ \prod_{k=1}^K z_{k,k}^{K - k + 1}
\right) 
\ = \
2^K
\, \prod_{k=1}^K z_{k,k}^{K-k+2}.
\]

Let $f^{-1}$ be the inverse transform from a $K + \binom{K}{2}$-vector
$y$ to the $K \times K$ covariance matrix $x$.  A density function
$p_X(x)$ defined on $K \times K$ covariance matrices is transformed to
the density $p_Y(y)$ over $K + \binom{K}{2}$ vectors $y$ by
\[
p_Y(y) = p_X(f^{-1}(y)) \ 2^K \ \prod_{k=1}^K z_{k,k}^{K-k+2}.
\]

\section{Cholesky Factors of Covariance Matrices}

An $M \times M$ covariance matrix $\Sigma$ can be Cholesky factored to
a lower triangular matrix $L$ such that $L\,L^{\top} = \Sigma$.  If
$\Sigma$ is positive definite, then $L$ will be $M \times M$.  If
$\Sigma$ is only positive semi-definite, then $L$ will be $M \times N$,
with $N < M$.  

A matrix is a Cholesky factor for a covariance matrix if and only if
it is lower triangular, the diagonal entries are positive, and $M \geq
N$.  A matrix satisfying these conditions ensures that $L \,
L^{\top}$ is positive semi-definite if $M > N$ and positive definite
if $M = N$.

A Cholesky factor of a covariance matrix requires $N + \binom{N}{2} +
(M - N)N$ unconstrained parameters.

\subsection{Cholesky Factor of Covariance Matrix Transform}

Stan's Cholesky factor transform only requires the first step of the
covariance matrix transform, namely log transforming the positive
diagonal elements.  Suppose $x$ is an $M \times N$ Cholesky factor.
The above-diagonal entries are zero, the diagonal entries are
positive, and the below-diagonal entries are unconstrained.  The
transform required is thus
%
\[
y_{m,n} = 
\left\{
\begin{array}{cl}
0 & \mbox{if } m < n,
\\[4pt]
\log x_{m,m} & \mbox{if } m = n, \mbox{ and}
\\[4pt]
x_{m,n} & \mbox{if } m > n.
\end{array}
\right.
\]

\subsection{Cholesky Factor of Covariance Matrix Inverse Transform}

The inverse transform need only invert the logarithm with an
exponentiation.  If $y$ is the unconstrained matrix representation,
then the elements of the constrained matrix $x$ is defined by
\[
x_{m,n} =
\left\{
\begin{array}{cl}
0 & \mbox{if } m < n,
\\[4pt]
\exp(y_{m,m}) & \mbox{if } m = n, \mbox{ and}
\\[4pt]
y_{m,n} & \mbox{if } m > n.
\end{array}
\right.
\]

\subsection{Absolute Jacobian Determinant of Cholesky Factor Inverse Transform}

The transform has a diagonal Jacobian matrix, the absolute determinant
of which is
%
\[
\prod_{n=1}^N  \frac{\partial}{\partial_{y_{n,n}}} \, \exp(y_{n,n})
\ = \ 
\prod_{n=1}^N \exp(y_{n,n})
\ = \
\prod_{n=1}^N x_{n,n}.
\]

Let $x = f^{-1}(y)$ be the inverse transform from a $N + \binom{N}{2}
+ (M - N)N$ vector to an $M \times N$ Cholesky factor for a covariance
matrix $x$ defined in the previous section.  A density function
$p_X(x)$ defined on $M \times N$ Cholesky factors of covariance
matrices is transformed to the density $p_Y(y)$ over $N + \binom{N}{2}
+ (M - N)N$ vectors $y$ by
%
\[
p_Y(y) = p_X(f^{-1}(y)) \prod_{N=1}^N x_{n,n}.
\]

\section{Cholesky Factors of Correlation Matrices}

A $K \times K$ correlation matrix $\Omega$ is positive definite and
has a unit diagonal.  Because it is positive definite, it can be
Cholesky factored to a $K \times K$ lower-triangular matrix $L$ with
positive diagonal elements such that $\Omega = L\,L^{\top}$.  Because
the corelation matrix has a unit diagonal,
\[
\Omega_{k,k} = L_k\,L_k^{\top} = 1, 
\]
each row vector $L_k$ of the Cholesky factor is of unit length.  The
length and positivity constraint allow the diagonal elemens of $L$ to
be calculated from the off-diagonal elements, so that a Cholesky
factor for a $K \times K$ correlation matrix requires only
$\binom{K}{2}$ unconstrained parameters.

\subsection{Cholesky Factor of Correlation Matrix Inverse Transform}

It is easiest to start with the inverse transform from the
$\binom{K}{2}$ unconstrained parameters $y$ to the $K \times K$
lower-triangular Cholesky factor $x$.  The inverse transform is based
on the hyperbolic tangent function, $\tanh$, which satisfies $\tanh(x)
\in (-1,1)$.  Here it will function like an inverse logit with a sign
to pick out the direction of an underlying canonical partial
correlation (see \refsection{correlation-matrix-transform} for more
information on the relation between canonical partial correlations and
the Cholesky factors of correlation matrices).

Suppose $y$ is a vector of $\binom{K}{2}$ unconstrained values.  Let
$z$ be a lower-triangular matrix with zero diagonal and below
diagonal entries filled by row.  For example, in the $3 \times 3$
case,
\[
z =
\left[
\begin{array}{ccc}
0 & 0 & 0
\\
\tanh y_1 & 0 & 0
\\
\tanh y_2 & \tanh y_3 & 0
\end{array}
\right] 
\]
%
The matrix $z$, with entries in the range $(-1,1)$, is then
transformed to the Cholesky factor $x$, by taking%
%
\footnote{For convenience, a summation with no terms, such as
  $\sum_{j' < 1} x_{i,j'}$, is defined to be 0.  This implies
  $x_{1,1} = 1$ and that $x_{i,1} = z_{i,1}$ for $i > 1$.}
%
\[
x_{i,j}
=
\left\{
\begin{array}{lll}
0 & \mbox{ if } i < j & \mbox{ [above diagonal]}
\\[12pt]
\sqrt{1 - \sum_{j' < j} x_{i,j'}^2}
  & \mbox{ if } i = j & \mbox{ [on diagonal]}
\\[12pt]
z_{i,j} \ \sqrt{1 - \sum_{j' < j} x_{i,j'}^2}
  & \mbox{ if } i > j & \mbox{ [below diagonal]}
\end{array}
\right.
\]
%
In the $3 \times 3$ case, this yields
\[
x = 
\left[
\begin{array}{ccc}
1 & 0 & 0
\\[6pt]
z_{2,1} & \sqrt{1 - x_{2,1}^2} & 0
\\[6pt]
z_{3,1} & z_{3,2} \sqrt{1 - x_{3,1}^2} 
        & \sqrt{1 - (x_{3,1}^2 + x_{3,2}^2)} 
\end{array}
\right],
\]
where the $z_{i,j} \in (-1,1)$ are the $\tanh$-transformed $y$.

The approach is a signed stick-breaking process on the quadratic
(Euclidean length) scale.  Starting from length 1 at $j=1$, each
below-diagonal entry $x_{i,j}$ is determined by the (signed) fraction
$z_{i,j}$ of the remaining length for the row that it consumes. The
diagonal entries $x_{i,i}$ get any leftover length from earlier
entries in their row.  The above-diagonal entries are zero.

\subsection{Cholesky Factor of Correlation Matrix Transform}

Suppose $x$ is a $K \times K$ Cholesky factor for some correlation
matrix.  The first step of the transform reconstructs the intermediate
values $z$ from $x$, 
\[
z_{i,j} = \frac{x_{i,j}}{\sqrt{1 - \sum_{j' < j}x_{i,j'}^2}}.
\]
%
The mapping from the resulting $z$ to $y$ inverts
$\tanh$, 
\[
y 
\ = \
\tanh^{-1} z
\ = \
\frac{1}{2} \left( \log (1 + z) - \log (1 - z) \right).
\]
%


\subsection{Absolute Jacobian Determinant of Inverse Transform}

The Jacobian of the full transform is the product of the Jacobians of
its component transforms.

First, for the inverse transform $z = \tanh y$, the derivative is
%
\[
\frac{d}{dy} \tanh y = \frac{1}{(\cosh y)^2}.
\]
%

Second, for the inverse transform of $z$ to $x$, the resulting
Jacobian matrix $J$ is of dimension $\binom{K}{2} \times
\binom{K}{2}$, with indexes $(i,j)$ for $(i > j)$.  The Jacobian
matrix is lower triangular, so that its determinant is the product of
its diagonal entries, of which there is one for each $(i,j)$ pair,
%
\[
\left| \, \mbox{det} \, J \, \right|
  \ = \ \mathlarger{\prod}_{i > j} \left| \frac{d}{dz_{i,j}} x_{i,j} \right|,
\]
%
where
%
\[
\frac{d}{dz_{i,j}} x_{i,j}
= \sqrt{1 - \sum_{j' < j} x^2_{i,j'}}.
\]
%
So the combined density for unconstrained $y$ is 
\[
p_Y(y) 
= p_X(f^{-1}(y)) 
  \ \
  \mathlarger{\prod}_{n < \binom{K}{2}} \frac{1}{(\cosh y)^2}
  \ \
  \mathlarger{\prod}_{i > j} \left( 1 - \sum_{j' < j} x_{i,j'}^2
  \right)^{1/2},
\]
%
where $x = f^{-1}(y)$ is used for notational convenience.  The log
Jacobian determinant of the complete inverse transform $x = f^{-1}(y)$
is given by
\[
\log \left| \, \det J \, \right|
= 
-2 \sum_{n \leq \binom{K}{2}}
\log \cosh y
\
+
\
\frac{1}{2} \
\sum_{i > j}
\log \left( 1 - \sum_{j' < j} x_{i,j'}^2 \right)
.
\]



